%-------------------------
% Resume in Latex
% Author : Md. Rayed Bin Wahed
% License : MIT
%------------------------

\documentclass[letterpaper,11pt]{article}

\usepackage{latexsym}
\usepackage[empty]{fullpage}
\usepackage{titlesec}
\usepackage{marvosym}
\usepackage[usenames,dvipsnames]{color}
\usepackage{verbatim}
\usepackage{enumitem}
%\usepackage[pdftex]{hyperref}
\usepackage{fancyhdr}
\usepackage[hidelinks]{hyperref}

\pagestyle{fancy}
\fancyhf{} % clear all header and footer fields
\fancyfoot{}
\renewcommand{\headrulewidth}{0pt}
\renewcommand{\footrulewidth}{0pt}

% Adjust margins
\addtolength{\oddsidemargin}{-0.5in}
\addtolength{\evensidemargin}{-0.5in}
\addtolength{\textwidth}{1in}
\addtolength{\topmargin}{-.5in}
\addtolength{\textheight}{1.0in}

\urlstyle{same}

\raggedbottom
\raggedright
\setlength{\tabcolsep}{0in}

% Sections formatting
\titleformat{\section}{
  \vspace{-4pt}\scshape\raggedright\large
}{}{0em}{}[\color{black}\titlerule \vspace{-5pt}]

%-------------------------
% Custom commands
\newcommand{\resumeItem}[2]{
  \item\small{
    \textbf{#1}{: #2 \vspace{-2pt}}
  }
}

\newcommand{\resumeSubheading}[4]{
  \vspace{-1pt}\item
    \begin{tabular*}{0.97\textwidth}{l@{\extracolsep{\fill}}r}
      \textbf{#1} & #2 \\
      \textit{\small#3} & \textit{\small #4} \\
    \end{tabular*}\vspace{-5pt}
}

\newcommand{\resumeSubItem}[2]{\resumeItem{#1}{#2}\vspace{-4pt}}

\renewcommand{\labelitemii}{$\circ$}

\newcommand{\resumeSubHeadingListStart}{\begin{itemize}[leftmargin=*]}
\newcommand{\resumeSubHeadingListEnd}{\end{itemize}}
\newcommand{\resumeItemListStart}{\begin{itemize}}
\newcommand{\resumeItemListEnd}{\end{itemize}\vspace{-5pt}}

%-------------------------------------------
%%%%%%  CV STARTS HERE  %%%%%%%%%%%%%%%%%%%%%%%%%%%%


\begin{document}

%----------HEADING-----------------
\begin{tabular*}{\textwidth}{l@{\extracolsep{\fill}}r}
  \textbf{\href{https://github.com/rayedbar/}{\Large Md. Rayed Bin Wahed}} & Email:  \href{mailto:rayedbinwahed@gmail.com}{rayedbinwahed@gmail.com}\\
  \href{https://github.com/rayedbar/}{Github Profile} & Mobile: +8801774882517\\
\end{tabular*}


%-----------EDUCATION-----------------
\section{Education}
  \resumeSubHeadingListStart
    \resumeSubheading
      {BRAC University}{Mohakhali, Dhaka}
      {Bachelor of Science in Computer Science;  CGPA: 3.75/4.00}{Jan 2012 -- Aug 2016}
  \resumeSubHeadingListEnd


%-----------EXPERIENCE-----------------
\section{Experience}
  \resumeSubHeadingListStart

    \resumeSubheading
      {Therap (BD) Ltd.}{Banani, Dhaka}
      {Software Engineer, System Architecture Team}{Oct 2016 - Present}
      \resumeItemListStart
        \resumeItem{About}
          {Therap (BD) Ltd. is a fully owned subsidiary of \href{https://www.therapservices.net/}{Therap Services LLC}, USA. Therap is an online documentation, reporting and communication software suite for agencies supporting individuals with Intellectual and Developmental Disabilities (I/DD).}
		\resumeItem{Core Responsibilities}{As a full-stack developer for the system team, I write and maintain modules mostly related to system administration and user login. Additionally, I write programs critical to the performance and security of the system. Finally, I authored front-end and back-end libraries that other teams use internally for their development.}        
        \resumeItem{Core Technologies}
          {Java EE, Spring, Hibernate, Javascript, SQL, Oracle Weblogic, Gradle, Git, Bash.}
      \resumeItemListEnd
      
   \resumeSubheading
     {BRAC University}{Mohakhali, Dhaka}
     {Teaching Assistant}{Jan 2014 - Aug 2016}
      \resumeItemListStart
       \resumeItem{Data Structures}{Taught Data Structures for a total of five semesters. I graded quizzes and assembled supplementary practice material for students}
       \resumeItem{Discrete Mathematics}{Taught Discrete Mathemetics for one semester grading quizzes and providing supplementary learning materials.}
      \resumeItemListEnd
        
  \resumeSubHeadingListEnd


%-----------Deep Learning-----------------
\section{Deep Learning}
  \resumeSubHeadingListStart
    \resumeSubheading
      {Degree Thesis}{}
      {Comparative Analysis Between Learning Models Using Facial Expression Recognition}{Jan 2016 - Oct 2016}
      \resumeItemListStart
      	\resumeItem{Objective}{Our objective was to investigate how efficiently Google's Inception-v3 deep neural network architecture performed in comparison to some popular variants of SVMs at the time, a few shallow neural nets and a CNN model designed by us in a computationally harsh environment such as mobile GPU.}
      	\resumeItem{Results}{We found that the Inception-v3 module outperformed all previous benchmarks and provided the best although actually testing the performance of each model in a mobile device was left as future work.}
      	\resumeItem{Link}{\href{https://drive.google.com/file/d/0B3oA6iIUeVM8ZUJzTllrZndGRGc/view?usp=sharing}{Comparative Analysis Between Inception-v3 And
Other Learning Systems Using Facial Expression Recognition}}
      \resumeItemListEnd
  \resumeSubheading
      {Deep Learning Specialization}{}
      {Five course Coursera specialization taught by Prof. Andrew Ng of Stanford Univsersity}{Dec 2016 - Mar 2017}
      \resumeItemListStart

      	\resumeItem{Course 1: Neural Networks and Deep Learning}{
			\begin{itemize}
				\item Understand the major technology trends driving Deep Learning
				\item Be able to build, train and apply fully connected deep neural networks
				\item Know how to implement efficient (vectorized) neural networks 
				\item Understand the key parameters in a neural network's architecture 
			\end{itemize}			      	
      	}
      	
      	\resumeItem{Course 2: Improving Deep Neural Networks, Hyperparameter tuning, Regularization and Optimization}{
			\begin{itemize}
				\item Understand industry best-practices for building deep learning applications.
				\item Be able to effectively use the common neural network "tricks", including initialization, L2 and dropout regularization, Batch normalization, gradient checking
				\item Be able to implement and apply a variety of optimization algorithms, such as mini-batch gradient descent, Momentum, RMSprop and Adam, and check for their convergence. 
				\item Understand new best-practices for the deep learning era of how to set up train/dev/test sets and analyze bias/variance
				\item Be able to implement a neural network in TensorFlow. 
			\end{itemize}
		}

		\resumeItem{Course 3: Structuring Machine Learning Projects}{
			\begin{itemize}
				\item Understand how to diagnose errors in a machine learning system, and 
				\item Be able to prioritize the most promising directions for reducing error
				\item Understand complex ML settings, such as mismatched training/test sets, and comparing to and/or surpassing human-level performance
				\item Know how to apply end-to-end learning, transfer learning, and multi-task learning
			\end{itemize}
		}
		
		\resumeItem{Course 4: Convolutional Neural Networks}{
			\begin{itemize}
				\item Understand how to build a convolutional neural network, including recent variations such as residual networks.
				\item Know how to apply convolutional networks to visual detection and recognition tasks.
				\item Know to use neural style transfer to generate art.
				\item Be able to apply these algorithms to a variety of image, video, and other 2D or 3D data.
			\end{itemize}
		}
		
		\resumeItem{Course 5: Sequence Models}{
			\begin{itemize}
				\item Understand how to build and train Recurrent Neural Networks (RNNs), and commonly-used variants such as GRUs and LSTMs.
				\item Be able to apply sequence models to natural language problems, including text synthesis. 
				\item Be able to apply sequence models to audio applications, including speech recognition and music synthesis.
			\end{itemize}
		}
      \resumeItemListEnd
  \resumeSubheading
  	   {Books}{}
  	   {Deep learning with Python by Francois Chollet of Google Brain}{}
  	   
  	   \resumeItemListStart
  	   
  	   
  	   \resumeItem{Review}{
		This book by the author of Keras (framework for TensorFlow) himself is a first class, hands on, deep dive into the essentials and best practices of Deep Learning. Working with real datasets, the book does a fantastic job explaining and implementing advanced concepts such as Transfer Learning, Text and Image Generation with Variable Autoencoders, and Generative Adversarial Networks (GANs)  	   
  	   }
       \resumeItemListEnd	
 \resumeSubHeadingListEnd


%--------PROGRAMMING SKILLS------------
\section{Programming Skills}
  \resumeSubHeadingListStart
  	\resumeSubItem{Languages}{Python, Keras, TensorFlow, Numpy, Pandas, Matplotlib, SciPy, Pytorch}
  \resumeSubHeadingListEnd


%-------------------------------------------
\end{document}
